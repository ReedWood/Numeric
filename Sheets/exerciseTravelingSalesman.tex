\documentclass[paper=a4, fontsize=12pt]{scrartcl}
\usepackage[utf8]{inputenc}
\usepackage[ngerman]{babel}
\usepackage{amsmath}

\pagestyle{empty}
\parindent0pt

\begin{document}

\begin{center}
{\Large{Numerische Methoden in der Physik}} \\
{\sc{Dr. Bj\"orn Schelter}} \\
\vspace{5mm}
{\large{Aufgabenblatt Nr. 4}} \\
\end{center}

\vspace{5mm}

{\bf Übung 4 \\ Stochastisches Minimieren: The Traveling Salesman}
\begin {itemize}
\item Lege Koordinaten (2D, (x,y)) für $N$ Städte fest.
\item Finde den kürzesten Weg, der alle Städte verbindet, ohne dass eine Stadt mehr als einmal
besucht wird. Verwende dazu \emph{Simulated Annealing}.
\begin{itemize}
\item Untersuche verschiedene Permutationsschemata
\item Untersuche verschiedene Abkühlungsschemata
\item Untersuchen verschiedene Anzahlen von Iterationsschritten.
\end{itemize}
\item[]
\item Gruppiere die Städte um einen Berg, und führe einen Strafterm ein, wann immer die Route über
den Berg geht.
\item Gehe über zu 3D-Koordinaten. Was ist der Unterschied zum Straftermansatz?
\end{itemize}

\end{document}
